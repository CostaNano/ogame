%%%%%%%%%%%%%%%%%%%%%%%%%%%%%%%%%%%%%%%%%
% Journal Article
% LaTeX Template
% Version 1.4 (15/5/16)
%
% This template has been downloaded from:
% http://www.LaTeXTemplates.com
%
% Original author:
% Frits Wenneker (http://www.howtotex.com) with extensive modifications by
% Vel (vel@LaTeXTemplates.com)
%
% License:
% CC BY-NC-SA 3.0 (http://creativecommons.org/licenses/by-nc-sa/3.0/)
%
%%%%%%%%%%%%%%%%%%%%%%%%%%%%%%%%%%%%%%%%%

%----------------------------------------------------------------------------------------
%	PACKAGES AND OTHER DOCUMENT CONFIGURATIONS
%----------------------------------------------------------------------------------------

\documentclass[twoside,twocolumn]{article}

\usepackage{blindtext} % Package to generate dummy text throughout this template

\usepackage[sc]{mathpazo} % Use the Palatino font
\usepackage[T1]{fontenc} % Use 8-bit encoding that has 256 glyphs
\linespread{1.05} % Line spacing - Palatino needs more space between lines
\usepackage{microtype} % Slightly tweak font spacing for aesthetics

\usepackage[english]{babel} % Language hyphenation and typographical rules

\usepackage[hmarginratio=1:1,top=32mm,columnsep=20pt]{geometry} % Document margins
\usepackage[hang, small,labelfont=bf,up,textfont=it,up]{caption} % Custom captions under/above floats in tables or figures
\usepackage{booktabs} % Horizontal rules in tables

\usepackage{lettrine} % The lettrine is the first enlarged letter at the beginning of the text

\usepackage{enumitem} % Customized lists
\setlist[itemize]{noitemsep} % Make itemize lists more compact

\usepackage{abstract} % Allows abstract customization
\renewcommand{\abstractnamefont}{\normalfont\bfseries} % Set the "Abstract" text to bold
\renewcommand{\abstracttextfont}{\normalfont\small\itshape} % Set the abstract itself to small italic text

\usepackage{titlesec} % Allows customization of titles
\renewcommand\thesection{\Roman{section}} % Roman numerals for the sections
\renewcommand\thesubsection{\roman{subsection}} % roman numerals for subsections
\titleformat{\section}[block]{\large\scshape\centering}{\thesection.}{1em}{} % Change the look of the section titles
\titleformat{\subsection}[block]{\large}{\thesubsection.}{1em}{} % Change the look of the section titles

\usepackage{fancyhdr} % Headers and footers
\pagestyle{fancy} % All pages have headers and footers
\fancyhead{} % Blank out the default header
\fancyfoot{} % Blank out the default footer
\fancyhead[C]{Running title $\bullet$ May 2016 $\bullet$ Vol. XXI, No. 1} % Custom header text
\fancyfoot[RO,LE]{\thepage} % Custom footer text

\usepackage{titling} % Customizing the title section

\usepackage{hyperref} % For hyperlinks in the PDF

%----------------------------------------------------------------------------------------
%	TITLE SECTION
%----------------------------------------------------------------------------------------

\setlength{\droptitle}{-4\baselineskip} % Move the title up

\pretitle{\begin{center}\Huge\bfseries} % Article title formatting
\posttitle{\end{center}} % Article title closing formatting
\title{Ogame Bot} % Article title
\author{%
\textsc{Jonathan Arndt}\\%\thanks{A thank you or further information} \\[1ex] % Your name
\normalsize Utah State University\\ % Your institution
\normalsize \href{mailto:unsupo@gmail.com}{unsupo@gmail.com} % Your email address
%\and % Uncomment if 2 authors are required, duplicate these 4 lines if more
%\textsc{Jane Smith}\thanks{Corresponding author} \\[1ex] % Second author's name
%\normalsize University of Utah \\ % Second author's institution
%\normalsize \href{mailto:jane@smith.com}{jane@smith.com} % Second author's email address
}
\date{\today} % Leave empty to omit a date
\renewcommand{\maketitlehookd}{%
\begin{abstract}
\noindent %\blindtext % Dummy abstract text - replace \blindtext with your abstract text
Ogame is a complicated strategy game which is known for banning players who use bots,
the bot created for this paper didn't not get discovered and banned and did achieve
the goal of getting 10s of thousands of point in a short time.

This bot still has a ways to go before it can truly conquor the universe, however
this is a difficult game that takes time and effort to truly be good at.
\end{abstract}
}

%----------------------------------------------------------------------------------------

\begin{document}

% Print the title
\maketitle

%----------------------------------------------------------------------------------------
%	ARTICLE CONTENTS
%----------------------------------------------------------------------------------------

\section{Introduction}
\lettrine[nindent=0em,lines=3]{O}game, the online strategy game is notorious
among it's community as both a long fund and tedious game as well as a brutal
banner of players who don't obey the rules.  Nonetheless, some developers
have made attempts to create bots to automate the tedious tasks involved in this
game.  Most bots have ceased development due to ogame upgrades and increases in
banning of players using them.

The most profitable of which is Ogame Automizer which charges players money to
bot their account.  This bot has ceased development since the latest version of
ogame has been released, yet still players on the forum ask for continued development.

This paper address my attempt to create a bot for this game and the challenges
and results of my efforts.

%------------------------------------------------

\section{Gameplay}
Ogame is an online multiplayer strategy game.
The game starts out with you selecting a universe and creating an account.
There are many universes and each one is completely seperated from each other.
Once you've created an account, you are given a planet to build onto.

The game revolves around the concept of upgrading and waiting an amount of time
for the upgrade to complete.  Upgrades falls into 3 categories: buildings, research
and ships.  Each category has a blocking item; an item to which another
of the same category can't be built. A time decreasing item; an item to which
the time it takes to improve the level of that category is decreased.  And lastly
an improvement tree; the tree by which you can improve items to a specified level
to unlock other items.

Resources are vital to the game, without which, the player literally can't do anything.
There are three main forms of resources: metal, crystal, and deueterium.  All
forms of improvements on all three categories require some proportion of these
three resources.  Each improvement means the next level will follow an
exponential curve of the next cost, usually by a power of 2.  Dark Matter could be
considered a resource, however this is a mostly real money paid resource and can
be gathered in game in minute quantities, this resource isn't needed to improve the
categories, but to decrease the time or decrease the resource gap needed for the
improvement.  Finally, energy, this resource is needed, but not consumed by a few
category improvements.  Energy's main point to power the producers, to which won't
function at top efficiency without a specified amount of energy available.

Ogame also has ways to generate resources, these will be called produces for this
paper.  The game doesn't have any built in resources (except for a tiny amount
supplied at the beggining of the game), resources can only be aquired by either
you improving producers, or by you raiding another player who has improved upon
producers (you cannot raid energy or dark matter from other players).  There are
also a few other ways to get resoures, all using ships.  This includes harvesting,
or getting resources from another player's planet's debris field after their, or
opposing player's ships have been destroyed and you can harvest a fraction of the
resources used to make the ships.  This method only allows harvesting of metal and
crystal.  Lastly, expodition, which is like the ingame lottory.  You can send ships
on an expedition to either lose all your ships, get some resources (including dark matter),
or get some ships.

Combat is an essential part of the game.  It fuels the competition and to some
extent the fun of the game.  Some players choose to live a solitary career, these
players are known as turtles, they simply build defenses and producers.  This
may be the sinmplest way to play, but it is also the slowest to advance due to
diminishing returns of your own producers.  By stealing or raiding other players
you can increase your resources by a tremendous amount.  However this means that
other players view you as a way to increase their resources.  Thus a method known
as fleet/resource saving was invented.  Fleet/resource saving is the method by
which using your ships to transfer all resources and ships off your planet if
you get attacked by another player.

Most gameplay is preordained by formulas, very few aspecs of the game is random.
This makes it a great game to simulate and automate.


%------------------------------------------------

\section{Goals}

The main goal of this project was to automate the entire game.  Selenium webframework
was the main tool used to accomplish this goal. Phantomjs allowed selenium to be run
on a headless browser.  Create a bot which could use an email address and
create a new user.  This user would then utilize algorithms to level up and conquor the
universe.  Twitter bots exists and this bot will have a few similar outcomes to the Twitter
bot in that it will need to bypass the internal bot detection system built into ogame.org.

Another goal was to create bots to work in tandem like a hivemind.  By creating
multiple bots to communicate together.

%------------------------------------------------

\section{Methods}

Selenium framework with the phantomjs browser was the tool to automate this web game.
Some http requests were mimicked to easily access data or perform some actions.
Data was then saved to a database.

The bot began by using a list of created email addresses using One.com.  It would then
create a user using one of these email addresses on ogame.org and then verify the email
address.

Once in the game, the bot would preceed to develop levels using a build simulation which
devised to optimal build strategy for a perticular goal.  The main initial goal of the
bot was to get to the point of being able to create small cargo ships.  Small cargos
are critical to gather more than the improved producers could create.

The attacking algorithm works as follows, initially get to small cargos, using verified
gift of dark matter, this could be achieved nearly immidiately in the start of the game.
Then using an api's request for players, find the nearest inactive player with low points
and blindly attack that player.  A blind attack is one that the bot is unsure of the outcome.
Then work to get espionage probes while maintaining a steady stream of attacks on that player,
which if the attack become successfuly is now considered a safe target.  If the attack
was unsucessfully meaning the player had defenses, then that player is marked on the do not
attack list in the database.

Once probes have been achieved, then probe all inactive players that are attackable.
Using the probes report you can determine if the players are safe targets, ie they have no
defense.  The probes can also tell you how much resources you can steal, if you don't
have enough cargos to get all the resources then develop more cargos.

This constant stream of attacks on players allows the bot to quickly rise in the ranks
until eventually it will get to the point of being able to get more advanced technology.

%------------------------------------------------

\section{Results}

The bot was successful in achieving all goals for a single bot.  The bot ran for months
and still hasn't been detected as a bot.  The bot achieved 10s of thousands of points
and millions of resources at it's disposal as well has hundreds of cargos without any
human intervention.


%------------------------------------------------

\section{Future}

The future holds a vast array of improvements to the bot including the fabled hive mind
of bots.  Unfortunately at this point, the bot only works on my ide thus a more user
friendly program would be very beneficial.

Other features of the game are still yet to be developed, like fleet/resource saving.
Right now this is untested and only works to save the fleet, resource movement is still
a feature to be added.  Multiple colonies are an important improvement as this would
massivly increase the resource output and reach of the empire.  Missiles are needed
to attack players with defenses and many other improvements to attack active players.

% \subsection{Subsection One}
%
% A statement requiring citation \cite{Figueredo:2009dg}.
% \blindtext % Dummy text
%
% \subsection{Subsection Two}
%
% \blindtext % Dummy text
%
% %----------------------------------------------------------------------------------------
% %	REFERENCE LIST
% %----------------------------------------------------------------------------------------
%
% \begin{thebibliography}{99} % Bibliography - this is intentionally simple in this template
%
% \bibitem[Figueredo and Wolf, 2009]{Figueredo:2009dg}
% Figueredo, A.~J. and Wolf, P. S.~A. (2009).
% \newblock Assortative pairing and life history strategy - a cross-cultural
%   study.
% \newblock {\em Human Nature}, 20:317--330.
%
% \end{thebibliography}

%----------------------------------------------------------------------------------------

\end{document}
